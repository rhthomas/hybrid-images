\documentclass[a4paper]{article}

% TODO Recalculate code start/end lines.
% TODO Fix FFT cock up.
% TODO Write iterative method.
% TODO Write fourier method.
% TODO Write about Gaussian filters.
% TODO Write about hybrid images.

\usepackage[margin=1in]{geometry}
\usepackage{graphicx}
\usepackage{parskip}
\usepackage[cache=false]{minted}

\setminted[python] {
    linenos,
    frame=single,
    fontsize=\footnotesize,
    tabsize=4, 
    breaklines
}

\title{\vspace{-5ex}COMP6223 Coursework 1\\Hybrid Images}
\date{Nov. 2018}
\author{Rhys Thomas, \tt{rt8g15@ecs.soton.ac.uk}}

\begin{document}
\maketitle

\section{Template Convolution}
Template convolution is a group operator used to modify pixels in an image. Methods of performing the convolution include either an iterative approach, walking through the image and multiplying/summing neighbouring pixels, or using the Fourier transform.
 
\subsection{Iterative Method}
The iterative method is a group operator that works by multiplying the kernel with every centre pixel in the image. This is quite computationally expensive since the number of floating point multiplications is given by equation~\ref{eq:iterative}.

% TODO Make this equation nicer.
\begin{equation}
    (Image_h-pad_h) \times (Image_w-pad_w) \times colours
    \label{eq:iterative}
\end{equation}

So for a $300\times 300$ tricolour image with a kernel size of $3\times 3$, this would result in 268,203 calculations! The source code for this iterative method is given below, however this introduces the requirement for more efficient methods.

\inputminted[
    firstline = 35,
    lastline = 62
]{python}{../hybrid.py}

\subsection{Fourier Method}
The Fourier convolution gives an equivalent to template convolution, except is more computationally efficient by working instead in the frequency domain.

\subsection{Gaussian Filters}
To low pass filter an image, convolve the Gaussian filter -- of particular cutoff frequency, $\sigma$ -- with the original image. One can simply subtract the low pass filtered result from the original image in order to produce the high pass filter equivalent.

\inputminted[
    firstline = 104,
    lastline = 126
]{python}{../hybrid.py}

\begin{figure}[!htbp]
    \centering
    \includegraphics[width=0.25\textwidth]{../low}
    \includegraphics[width=0.25\textwidth]{../high}
    \caption{Low and high pass filtered images with $\sigma=5$.}
    \label{fig:high-low}
\end{figure}

Notice in figure~\ref{fig:high-low} the boarder around the images. This is padding due to the iterative convolution method, and would not otherwise be present in the equivalent operation using the Fourier method.

\section{Hybrid Images}
\begin{figure}[!htbp]
    \centering
    \includegraphics[width=0.5\textwidth]{../visual}
    \caption{Visualising the effect of hybrid images filtered with $\sigma=5$.}
    \label{fig:visual}
\end{figure}

\section{Usage}
The \texttt{argparse} python module was used to generate arguments for the script. It allows the user to define kernel sizes -- for applying arbitrary sized Sobel templates for example -- low/high cutoff frequencies for the Gaussian filter, input images to process and output destinations for the hybrid result and the visualisation.

\begin{minted}[
    frame=single,
    fontsize=\footnotesize,
    tabsize=4
]{text}
usage: hybrid.py [-h] -i IMAGE IMAGE [-k KERNEL KERNEL] [-c CUTOFF CUTOFF] -o
                 OUTPUT -v VISUAL

optional arguments:
  -h, --help            show this help message and exit
  -i IMAGE IMAGE, --image IMAGE IMAGE
                        Path to input images.
  -k KERNEL KERNEL, --kernel KERNEL KERNEL
                        Kernal size, e.g. 5 7. Note: first image in list will
                        be used.
  -c CUTOFF CUTOFF, --cutoff CUTOFF CUTOFF
                        Gaussian cutoff frequencies, e.g. 5 5.
  -o OUTPUT, --output OUTPUT
                        Path to output image file.
  -v VISUAL, --visual VISUAL
                        Path to output visualisation file.    
\end{minted}

Example usage is shown below, with the convolution taking place between ``dog.bmp'' and ``cat.bmp'' with the cutoff frequency of $\sigma=4$ for both low and high pass. The output hybrid image is saved to ``hybrid.jpg'' and the visualisation is saved to ``visual.jpg''.
\begin{minted}[
    frame=single,
    fontsize=\footnotesize,
    tabsize=4
]{text}
> ./hybrid.py -i data/dog.bmp data/cat.bmp -c 4 4 -o hybrid.jpg -v visual.jpg
[data/dog.bmp]	Generating low pass image...
[data/dog.bmp]	Calculating Gaussian kernel...
[data/cat.bmp]	Generating high pass image...
[data/cat.bmp]	Calculating Gaussian kernel...
Creating hybrid image...
Creating visualisation...
Done.   
\end{minted}

\end{document}